%!TEX root = main.tex
\chapter{Problems, Issues and Points Learned}
\label{cha:problems_issues_and_points_learned}

% In addition to listing problems and issues with the document or with the implementation process, this is also a spot to reflect upon the project and discuss what you would have done differently if you were to start again from scratch.


We decided to keep the game as simple as possible, but keep open possibilities for extensions if time allowed. This is some of the reason that we decided to implement a rather large amount of patterns. While we managed to implement most of them as per our specifications, we have realised that we should have been more cautious and only implemented a few as properly as possible instead.

Either way, we all feel that we have learned a lot from this project, both from the experience of implementing the patterns, and from having to work together as a team.


\section{Human resources}
The main issue throughout the duration of the project is human resources, with only a few of the persons involved doing most of the work. 


\subsection{Communication issues}
While both an IRC channel and an email list was used in the course of this project, the information flow was extremely lacking. Some only got important messages if they were posted on the IRC channel, others only when the information was posted on the mailing list. Important information got lost, which made planning and information exchange difficult.


\subsection{Lack of participation}
Most of the work on the project was done by a fraction of the group, even when the group actually was able to properly meet. Only a few of us did work outside meeting hours.


\section{Time constraints}
The time allotted for this project was rather limited, and as a result some game mechanics had to be omitted or given a low priority.


\subsection{Artifical Intelligence}
Large and Heavily Armoured Warships is a multiplayer game at its core. To limit the scope of the game, we did not intend to implement the possibility to play the game over a network connection, but rather implement a simple artificial intelligence for the player to play against. For a simple game such as LaHAW, this could be limited to a glorified random number generator with some heuristics to better know where the enemy ships are. Unfortunately, the development of these heuristics had to be omitted, and only a simple random number generator remains. This makes the game (statistically) impossible to lose, as the AI will try to bomb a tile that it already has bombed at some point in the game which the human player probably won't do.

\subsection{Help system}
To aid the players of LaHAW in learning and understand the game play, a help system was to be implemented. The help system implemented is limited to a help button on the game's main menu, and does not contain any information. A proper help tool would contain actual help.

\subsection{Customisation}
The player should per the requirement documentation be able to set player name as well as the colour of the boats. While the game stores this information on the device, only the player name is actually used. While the intention was to paint the ships with the player colour, the game now only paints the player name in the set colour.



\section{Code}
\subsection{Naming conventions}
\label{sec:naming_conventions}
At the very beginning of the implementation of LaHAW, the group decided upon a naming convention that later proved to make the code incredibly hard to read and write.

An example of how the code was written and modified is the Warship class. MVC dictates that every object in the application code should have its own model, its own view and its own controller. Our initial naming scheme consisted of using packages to distinguish between the different objects{\footnote{\texttt{model/Warship.java}, \texttt{view/Warship.java} and \texttt{controller/Warship.java}}}. While clever, this had the effect of making the code incredibly hard to read and write. % Kanskje skrive mer direkte om konsekvensene.
We had to decide upon a different scheme, where the class name distinguished between the objects \footnote{\texttt{model/WarshipModel.java}, \texttt{view/WarshipView.java} and \texttt{controller/WarshipControl- ler.java}}.
