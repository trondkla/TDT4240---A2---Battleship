%!TEX root = main.tex
\chapter{Test Report}

%The report should contain test reports for both functional requirements and quality requirements (quality scenarios).

This chapter will detail the test reports for both the functional requirements found in % Todo: legg inn referanse
, and our quality requirements found in % Todo: legg inn referanse.


%-----------------------------------------
\section{Functional requirements}
%-----------------------------------------

\begin{tabular}{|p{2cm}|p{9.5cm}|}
	\hline
	\bf{Type}	& \bf{Info} \\
	\hline
	FR1			& Change difficulty/size of ocean space \\
	Tester		& Perry \\
	Executor	& Bremnes \\
	Date		& \date{\today} \\
	Evaluation	& Possible to change ocean space size, but the AI and the player have the same ocean space size. The game is equally difficult for the AI and the player. \\
	\hline
\end{tabular}

\vspace{0.5em}

\noindent
\begin{tabular}{|p{2cm}|p{9.5cm}|}
	\hline
	\bf{Type}	& \bf{Info} \\
	\hline
	FR2			& Set/change player name \\
	Tester		& Perry \\
	Executor	& Bremnes \\
	Date		& \date{\today} \\
	Evaluation	& Changing name before and between game sessions works. \\
	\hline
\end{tabular}

\vspace{0.5em}

\noindent
\begin{tabular}{|p{2cm}|p{9.5cm}|}
	\hline
	\bf{Type}	& \bf{Info} \\
	\hline
	FR3			& Game over \\
	Tester		& Perry \\
	Executor	& Bremnes \\
	Date		& \date{\today} \\
	Evaluation	& After each game session, the game ends, and a game over state is shown. \\
	\hline
\end{tabular}

\vspace{0.5em}

\noindent
\begin{tabular}{|p{2cm}|p{9.5cm}|}
	\hline
	\bf{Type}	& \bf{Info} \\
	\hline
	FR4			& Place ones ships at the start of the game \\
	Tester		& Perry \\
	Executor	& Bremnes \\
	Date		& \date{\today} \\
	Evaluation	& By draggin and droppin a ship, it is possible for the player to position them. By simulanously tapping the screen while dragging, the ship can be rotated. \\
	\hline
\end{tabular}

\vspace{0.5em}

\noindent
\begin{tabular}{|p{2cm}|p{9.5cm}|}
	\hline
	\bf{Type}	& \bf{Info} \\
	\hline
	FR5			& Play audio \\
	Tester		& Perry \\
	Executor	& Bremnes \\
	Date		& \date{\today} \\
	Evaluation	& Audio is played when player hits the AI player. \\
	\hline
\end{tabular}

\vspace{0.5em}

\noindent
\begin{tabular}{|p{2cm}|p{9.5cm}|}
	\hline
	\bf{Type}	& \bf{Info} \\
	\hline
	FR6			& Hit enemy ships \\
	Tester		& Perry \\
	Executor	& Bremnes \\
	Date		& \date{\today} \\
	Evaluation	& When a player bomb their opponent's ship, a static explosion is shown at the bombed ocean tile. \\
	\hline
\end{tabular}

\vspace{0.5em}

\noindent
\begin{tabular}{|p{2cm}|p{9.5cm}|}
	\hline
	\bf{Type}	& \bf{Info} \\
	\hline
	FR7			& Register hits on friendly ships \\
	Tester		& Perry \\
	Executor	& Bremnes \\
	Date		& \date{\today} \\
	Evaluation	& The AI fires randomly at the player's ocean space, and each hit is marked with the same static explosion as mentioned earlier. \\
	\hline
\end{tabular}

\vspace{0.5em}


	%-----------------------------------------
	\subsection{Conclusion}
	%-----------------------------------------
	The difficulty levels as per FR1 are not implemented propery. While the action to change the difficulty level does impact ones ocean space, this setting is not asymmetrical as the requirement requires. This results in the setting not impacting actual difficulty level. A proper implementation would set this setting asymmetrically.
	It is not clear what ocean space should be affected by the setting, and as it is worded now could equally mean "I want an easy game" and "I want my ocean space to be easy for my opponent". This has to be amended. This issue is discussed in further detail in section \ref{cha:problems_issues_and_points_learned}.

	The placement of the ships as per FR4 is implemented, but the user interface for rotating ships is poor.

	The implementation of the articifial intelligence is lacking intelligence, and it is statistically impossible to lose a game against the computer. While a functional AI was not a part of our functional requirements, this could potentially have a negative effect on the game's replayability. The AI also has a tendency to copy the player's ship placements, which can make the game even easier to win.

	The prototype functions adequately in accordance to our functional requirements. While there are some issues, a player can play and finish a complete game og LaHAW.



%-----------------------------------------
\section{Quality requirements}
%-----------------------------------------
\noindent
\begin{tabular}{|p{3cm}|p{8.5cm}|}
	\hline
	\bf{Type}	& \bf{Info} \\
	\hline
	M1			& Change difficulty \\
	Executor	&  \\
	Date		& \date{\today} \\
	Stimulus	& Add difficulty opportunity \\
	Expected response & 10 hours to implement and test \\
	Observed response & \\
	Evaluation	& \\
	\hline
\end{tabular}

\vspace{0.5em}

\noindent
\begin{tabular}{|p{3cm}|p{8.5cm}|}
	\hline
	\bf{Type}	& \bf{Info} \\
	\hline
	M2			& Set player color \\
	Executor	&  \\
	Date		& \date{\today} \\
	Stimulus	& Add opportunity to change your player colour \\
	Expected response & 10 hours to implement and test\\
	Observed response & \\
	Evaluation	& \\
	\hline
\end{tabular}

\vspace{0.5em}

\noindent
\begin{tabular}{|p{3cm}|p{8.5cm}|}
	\hline
	\bf{Type}	& \bf{Info} \\
	\hline
	M3			& Set player name \\
	Executor	&  \\
	Date		& \date{\today} \\
	Stimulus	& Add opportunity to set your player name \\
	Expected response & 5 hours to implement and test\\
	Observed response & \\
	Evaluation	&  \\
	\hline
\end{tabular}

\vspace{0.5em}

\noindent
\begin{tabular}{|p{3cm}|p{8.5cm}|}
	\hline
	\bf{Type}	& \bf{Info} \\
	\hline
	U1			& Placing the ships \\
	Executor	&  \\
	Date		& \date{\today} \\
	Stimulus	& Change position of the ships \\
	Expected response & The system shall prevent the user from trying to place a ship in an illegal position.\\
	Observed response & \\
	Evaluation	&  \\
	\hline
\end{tabular}

\vspace{0.5em}

\noindent
\begin{tabular}{|p{3cm}|p{8.5cm}|}
	\hline
	\bf{Type}	& \bf{Info} \\
	\hline
	U2			& Save player name and colour \\
	Executor	&  \\
	Date		& \date{\today} \\
	Stimulus	& Store player name and color \\
	Expected response & The player name will appear at the game play screen, \\
             & and the ships will be painted in the predefined player color  \\
	Observed response & \\
	Evaluation	&  \\
	\hline
\end{tabular}

\vspace{0.5em}

\noindent
\begin{tabular}{|p{3cm}|p{8.5cm}|}
	\hline
	\bf{Type}	& \bf{Info} \\
	\hline
	U3			& Dialog boxes confirming the users actions \\
	Executor	&  \\
	Date		& \date{\today} \\
	Stimulus	& User regret their decision \\
	Expected response & Dialog box with confirmation button(s) (i.e. "OK")\\
	Evaluation	&  \\
	\hline
\end{tabular}

\vspace{0.5em}

\noindent
\begin{tabular}{|p{3cm}|p{8.5cm}|}
	\hline
	\bf{Type}	& \bf{Info} \\
	\hline
	U4			& No action should be more than three clicks away \\
	Executor	&  \\
	Date		& \date{\today} \\
	Stimulus	& User wants to use the system efficiently \\
	Expected response &  \\
	Evaluation	&  \\
	\hline
\end{tabular}

\vspace{0.5em}

\noindent
\begin{tabular}{|p{3cm}|p{8.5cm}|}
	\hline
	\bf{Type}	& \bf{Info} \\
	\hline
	U5			& The game should show game hints whereever appropriate \\
	Executor	&  \\
	Date		& \date{\today} \\
	Stimulus	& User is uncertain on how the game is played, \\
             & or what their next move should be \\
	Expected response & User can play the game without problem regarding to game control in \emph{1 hour}\\
	Observed response & \\
	Evaluation	&  \\
	\hline
\end{tabular}


	%-----------------------------------------
	\subsection{Conclusion}
	%-----------------------------------------

	The hint systen as per U3 requires the game to have redundant dialog boxes, is forcing some of the actions to have additional steps. This is in direct opposition to requirement U4 demanding no action being more than threee clicks away. Sparse use of this has however accomplished both requirements in a satifactory manner.

	The hint system is implemented as per U5, but does not include any actual hints. This makes the evaluation of the effect of such a system hard. The game thus fails to accomplish this requirement.

	While LaHAW implements the quality requirements defined with a somewhat varying degree of success, we conclude with that LaHAW passes the tests in a satisfactory manner.
